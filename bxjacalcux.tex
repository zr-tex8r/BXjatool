% 文字コードは UTF-8
% lualatex で組版する(LuaTeX-ja 使用)
\documentclass[a4paper]{ltjsarticle}
\usepackage[ipaex]{luatexja-preset}
\usepackage{metalogo}
\usepackage{shortvrb}
\MakeShortVerb{\|}
\newcommand{\PkgVersion}{0.2a}
\newcommand{\Pkg}[1]{\textsf{#1}}
\newcommand{\Meta}[1]{$\langle$\mbox{}#1\mbox{}$\rangle$}
\newcommand{\Note}{\par\noindent ※}
\newcommand{\Means}{~:\quad}
\providecommand{\pTeX}{p\TeX}
\providecommand{\upTeX}{u\pTeX}
%-----------------------------------------------------------
\begin{document}
\title{\Pkg{bxjacalcux} パッケージ(v\PkgVersion)}
\author{八登崇之\ (Takayuki YATO; aka.~``ZR'')}
\date{2013/05/05}
\maketitle

%===========================================================
\section{概要}

{\pTeX}系エンジンの独自拡張である |zw| や |Q| 等の長さ単位を
他のエンジン上の{\LaTeX}でも使用可能にする。

\paragraph{対応フォーマット} \LaTeX。
\paragraph{対応エンジン} 全て。
ただし日本語対応環境以外では機能制限がある。

\paragraph{依存パッケージ}
\begin{itemize}
\item 本パッケージと同じバンドルに含まれる
\Pkg{bxcalcize}、\Pkg{bxcalcux}。
\item \Pkg{bxtoolbox} パッケージ(\Pkg{BXbase}バンドルに含まれる)
\item \Pkg{calc}パッケージ
\end{itemize}

%===========================================================
\section{パッケージの読込}

|\usepackage| で読み込む。
オプションはない。
\begin{verbatim}
\usepackage{bxjacalcux}
\end{verbatim}

%===========================================================
\section{機能}

{\LaTeX}において長さを指定するほとんどの箇所
\footnote{正確には「\Pkg{calc}パッケージの数式が使用可能な箇所」である。
ただし、\Pkg{bxcalcize}パッケージで拡張されているので
標準の{\LaTeX}命令についてはほぼ完全対応だと思われる。
\Pkg{TikZ}パッケージのように独自の数式解析器ライブラリを
用いているものは残念ながら対象外となる。}
において以下にあげる「{\pTeX}独自の長さ単位」が使えるようになる。
\begin{itemize}
\item |Q|、|H|: |0.25mm| に等しい。
\item |trueQ|、|trueH|: |0.25truemm| に等しい。
\item |zw|、|zh|: 以下の環境でのみ使用可能。
\begin{itemize}
\item ({\pTeX}系では元々使用可能である。)
\item {\LuaTeX}-jaを使用する場合は、各々 |\zw|、|\zh| と等しくなる。
\item \Pkg{BXjscls}バンドルの文書クラスを使用する場合は、
ともに |\jsZw| と等しくなる。
\end{itemize}
\end{itemize}

%===========================================================
\section{詳細説明}

このパッケージは以下のように動作している。
\begin{enumerate}
\item \Pkg{calc}、\Pkg{bxcalcize}、\Pkg{bxcalcux}パッケージを読み込む。
\item 前述の「{\pTeX}の単位」を\Pkg{bxcalcux}の|\DeclareCalcUnit|命令を
利用して定義する。
\end{enumerate}

なお、エンジンが{\pTeX}系である場合は 2 の動作は行わないが、
他のエンジンとの挙動を合わせるため 1 は行っている。

\end{document}
