% 文字コードは UTF-8
% lualatex で組版する(LuaTeX-ja 使用)
\documentclass[a4paper]{ltjsarticle}
\usepackage[ipaex]{luatexja-preset}
\usepackage{metalogo}
\usepackage{shortvrb}
\MakeShortVerb{\|}
\newcommand{\PkgVersion}{0.3a}
\newcommand{\Pkg}[1]{\textsf{#1}}
\newcommand{\Meta}[1]{$\langle$\mbox{}#1\mbox{}$\rangle$}
\newcommand{\Note}{\par\noindent ※}
\newcommand{\Means}{~:\quad}
\providecommand{\pTeX}{p\TeX}
\providecommand{\upTeX}{u\pTeX}
%-----------------------------------------------------------
\begin{document}
\title{\Pkg{bxjaprnind} パッケージ(v\PkgVersion)}
\author{八登崇之\ (Takayuki YATO; aka.~``ZR'')}
\date{2013/05/05}
\maketitle

%===========================================================
\section{概要}

以下の機能を提供する。
\begin{itemize}
\item 段落冒頭および強制改行後の行頭における開き括弧の前に
  (環境により)挿入されるグルーを除去する。
  このようなグルーが行分割の都合で伸縮すると、
  ソース上で同じ条件にある括弧の位置が揃わないという
  不都合が生じる。
\item さらに、そのような開き括弧の前の字下げを、
  著者が予め設定した量に自動的に調整する。
\item 強制改行命令(|\\|)のオプションとして
  改行後に予め設定した量の字下げを行えるようにする。
\end{itemize}

\paragraph{対応フォーマット} \LaTeX。

\paragraph{対応エンジン} 次の何れか。
\begin{itemize}
\item {\pTeX}またはその拡張。
\item {\XeTeX}(\Pkg{ZXjatype}パッケージ使用時)
\item {\LuaTeX}(\Pkg{LuaTeX-ja}パッケージ使用時)
\end{itemize}

\paragraph{依存パッケージ} 次の通り。
\begin{itemize}
\item \Pkg{bxtoolbox} パッケージ(\Pkg{BXbase}バンドルに含まれる)
\item 0.3版以降は\Pkg{everyhook}パッケージに依存しなくなった。
しかし、\Pkg{everyhook}パッケージが利用可能である場合はそれを利用する。
\item \Pkg{everyhook}パッケージは\Pkg{svn-proc} パッケージに依存する。
\end{itemize}

%===========================================================
\section{パッケージの読込}

|\usepackage| で読み込む。
オプションはない。
\begin{verbatim}
\usepackage{bxjaprnind}
\end{verbatim}

%===========================================================
\section{機能}

以下の命令が提供される。

\begin{itemize}
\item |\useparheadparenindent|/|\nouseparheadparenindent|\Means
  段落頭での開き括弧調整を有効/無効にする。
\item |\uselineheadparenindent|/|\nouselineheadparenindent|\Means
  強制改行後の行頭での開き括弧調整を有効/無効にする。
  \Note 折り返しの行頭の調整には対応しない
  (これにはJFMの修正が必須)。
  現在(u){\pTeX}で標準的に用いられるJFMの場合、
  折り返し行頭の開き括弧の前には空きは入らない。
\item |\prnind[|\Meta{実数$r$}|]|\Means
  段落頭で用いて、当該の箇所での調整を
  (|\nouseparheadparenindent| の状態でも)有効にする。
  オプションの引数が存在する場合は、
  開き括弧の前の空きを$r$全角幅とする。
\item |\\[|\Meta{引数}|]|/|\\*[|\Meta{引数}|]|\Means
  強制改行命令のオプション引数が次のように拡張される。
  \begin{itemize}
  \item |\\[>]|\Means
    行頭に |\lineheadforceindentamount| で指定した幅の
    字下げを挿入する。
  \item |\\[!]|\Means
    開き括弧調整の有効・無効を逆転する。
    元々の引数(改行調整の値)と併用する場合は、
    |!| や |>| を長さの値の前に記述する
    (例えば |\\[>!2mm]|)。
  \end{itemize}
\item |\parheadparenindentamount{|\Meta{実数$r$}|}|\Means
  段落頭での開き括弧の前の追加の空きを$r$全角幅とする。
  段落下げの空き(|\parindent|)はこれとは独立に入ることに注意。
\item |\lineheadparenindentamount{|\Meta{実数$r$}|}|\Means
  強制改行後の行頭での開き括弧の前の追加の空きを$r$全角幅とする。
  (強制字下げ(|\\[>]|)の空きとは独立。)
\item |\lineheadforceindentamount{|\Meta{実数$r$}|}|\Means
  強制字下げ(|\\[>]|)の空きを$r$全角幅とする。
\item |\usedialogueparenindent|/|\nousedialogueparenindent|\Means
  会話用の特別な開き括弧調整を有効/無効にする。
  有効にすると鉤括弧「」『』について以下のように調整方法が変わる。
  \begin{itemize}
  \item 空きの量を |\dialogueparenindentamount| で指定された値にする。
  \item 段落頭の場合は段落下げの空き(|\parindent|)を無効にする。
  \end{itemize}
  \Note
  臨時に鉤括弧の扱いを通常の括弧と同じにしたい場合は、
  当該の鉤括弧の前に |\prnind| を置けばよい。
\item |\dialogueparenindentamount{|\Meta{実数$r$}|}|\Means
  段落頭・強制改行行頭での会話用の開き鉤括弧の前の追加の
  空きを$r$全角幅とする。
\end{itemize}

\end{document}
