\documentclass[a4paper,uplatex]{jsarticle}
\usepackage{bxjaprnind}
\newcommand{\Pkg}[1]{\textsf{#1}}
\begin{document}
\title{\Pkg{bxjaprnind}パッケージサンプル出力}
\author{八登崇之\ (Takayuki YATO; aka.~``ZR'')}
\date{2013/04/29}
\maketitle

%-------------------
\section{段落頭の括弧類の空き自動補正}
\newcommand\SampleTextA{%
「□□□□?」\par
「□□□□□□□□、□□□□□□□□」\par
「□□□□□□□□!」\par
}

\begin{itemize}
\item 左は自動補正を無効化(\verb|\nouseparheadparenindent|)。
\item 右は自動補正を有効化(\verb|\useparheadparenindent|)。
\end{itemize}
\begin{center}
% 既定は有効だが敢えて無効にする
\nouseparheadparenindent   % 段落頭での補正を無効に
\fbox{\begin{minipage}{10zw}
  \setlength{\parindent}{1zw}
  \SampleTextA
\end{minipage}}%
\hspace{4zw}
\useparheadparenindent   % 段落頭での補正を有効に
\fbox{\begin{minipage}{10zw}
  \setlength{\parindent}{1zw}
  \SampleTextA
\end{minipage}}%
\end{center}

%-------------------
\section{段落頭の括弧類の空きの補正値の変更}
\newcommand\SampleTextB{%
「□□□□□□□、□□□□□□□□□□□□□□□」
}

\begin{itemize}
\item 左は既定(\verb|\parheadparenindentamount{0}|)。
\item 右は半角下げに変更(\verb|\parheadparenindentamount{-0.5}|)。
\item 段落下げ(\verb|\parindent|)が入った位置を基準にしている。
\item 強制改行後の空白の調整は可能(\verb|\lineheadparenindentamount|)
だが折り返し行頭の空白の調整は{\pTeX}の仕様上困難である。
\end{itemize}
\begin{center}
\parheadparenindentamount{0} % 全角下げ(既定値)
\fbox{\begin{minipage}{10zw}
  \setlength{\parindent}{1zw}
  \SampleTextB
\end{minipage}}%
\hspace{4zw}
\parheadparenindentamount{-0.5} % 二分下げ
\fbox{\begin{minipage}{10zw}
  \setlength{\parindent}{1zw}
  \SampleTextB
\end{minipage}}%
\end{center}

%-------------------
\section{強制改行後の括弧類の空きの補正値の変更}
\newcommand\SampleTextC{%
「□□□□?」\\
「□□□。□□!」\\
「□□□□□□。□□□□□□□□□□□□□□」\\
「□□□□□□□!」
}

\begin{itemize}
\item ソースでは各台詞の間に強制改行が入れられている。
\item 左は既定の設定。
\item 右は %
\verb|\parheadparenindentamount{-0.5}\lineheadparenindentamount{0.5}|。
\item 鉤括弧以外の括弧にも効くのでグローバル設定には不適切であるが、
この設定にする環境を作る等の方法が考えられる。
\end{itemize}
\begin{center}
\fbox{\begin{minipage}{10zw}
  \setlength{\parindent}{1zw}
  \SampleTextC
\end{minipage}}%
\hspace{4zw}
\parheadparenindentamount{-0.5}\lineheadparenindentamount{0.5}
\fbox{\begin{minipage}{10zw}
  \setlength{\parindent}{1zw}
  \SampleTextC
\end{minipage}}%
\end{center}



\end{document}

